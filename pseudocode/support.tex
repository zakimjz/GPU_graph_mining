\section{Computation of support}
%\noindent Let have two graphs $\G=(V,E), \G'=(V',E')$ such that
%$\G\subseteq \G'$. 

\textbf{A sequence of nodes for recording the dfs code is fine but
  missing is the first node in ${\cal Q}$!}

\emph{gaston-like representation using arrays:} let have a pattern
$\pattern$. The representation of embeddings $\bigcup_{\G'\in\db}
\embed(\pattern, \G')$ can be done be using vector ${\cal Q} =
(Q_0,Q_1,\ldots,Q_n), n = |\mindfscode(\G)|$ where for $i\geq 1$
each $Q_i = ((idx_{1}, vid_{1}, gid_{1}), \ldots, (idx_{\ell},
vid_{\ell}, gid_{\ell}))$ is a vector and $idx_j$ corresponds to an
index into $Q_{i-1}$, $gid_j$ corresponds to the index of the graph
$\G_{gid_j}\in\db$, and $vid_j\in V$.  The first element of
${\cal Q}$, the $Q_0$, is special: it is the first node of the
embedding. $Q_0$ therefore has initialized only $vid_j$ and
$gid_j$. The $idx_j$ of $Q_0$ is uninitialized or set to $-1$. The
elemens $Q_1, \ldots, Q_n$ of ${\cal Q}$ are associated with
$\mindfscode(G)=(c_1,\ldots, c_{|E|})$ that describes the columns,
i.e., $i$th column $Q_i$ corresponds to $c_i$. This representation
should have smaller size in the memory then the naive representation
(see later), but the memory access seems to be worse then in the
previous case.

\emph{Naive representation of embeddings:} let have a pattern
$\pattern$. The representation of the embeddings
$\bigcup_{\G'\in\db}\embed(\pattern,\G')$ can be done using a
vector ${\cal Q}=(Q_0, Q_1,\ldots, Q_n)$ of size $n$ and each $Q_i$ of
size $m=|\bigcup_{\G'\in\db} \embed(\pattern, \G')|$, i.e.,
${\cal Q}$ corresponds to a $n\times m$ matrix. We omit the structure
of each $Q_i$, it is similar to the gaston-like representation.  We
will keep the representation as a vector of vectors for practical
reasons: this representation of matrix allow us to describe both
representation of embedings (naive+gaston).

\begin{definition}
\textbf{still may be incorrect}
Let have a pattern $\pattern = (V_\pattern, E_\pattern)$ with
canonical code $\mindfscode(\pattern) = (c_1, \ldots,
c_{|E_\pattern|})$ and the right-most path $\rmpath$ associated with
$\mindfscode(P)$. Let $V_\rmpath$ are all the nodes in the pattern on
the right-most path. Let have a graph $G=(V,E)$. Let
$\G'=(V',E')\subseteq \G$ be one embedding of $\pattern$. Let
the function $f$ maps nodes from $V'$ into $V_\pattern$.

\begin{enumerate}
\item Backward extensions of one such embedings are the 5-tuples
  $(f(v_i), f(v_j), l_i, l_{ij}, l_j)$ such that $f(v_i), f(v_j)$ is on the
  right-most path of $\mindfscode(P)$.
\item Forward extensions are the 5-tuples $(f(v_i), v_{|V|+1}, l_i,
  l_{ij}, l_j)$, such that $f(v_i)$ is on the right-most path and
  $v_{|V|+1}\not\in V'$
\end{enumerate}

\noindent We will denote the extensions for one graph $G$ by $\extend(\pattern,
\G)$. The extensions of $\mindfscode(\pattern)$ in a database $\db$ is the
union of extensions $\bigcup_{\G\in\db} \extend(\pattern, \G)$,
denoted by $\extend(\pattern, \db)$.


%\noindent The inverse function $f^{-1}$ returns a set of vertices. Then the
%frontier of the pattern $\pattern$ in the database $\db$ is
%$\bigcup\{f^{-1}(v') | X=(v, v', \in \extend(\pattern, \db), \text{X is
%  forward}\}$

\end{definition}

It does not make sense to define the frontier in the same fasion as
for the BFS search of a graph. Rather, we use the notion of extensions
of a pattern. The frontier is in fact part of ${\cal Q}$, i.e., the
$Q_i$ that corresponds to the canonical code elements on the
right-most path.

Discussion: instead of having $gid$ in the vectors $Q_i$, we can
assign each node of the graph $G\in \db$ a unique id. However the
tuple $(vid,gid)$ represent such a unique id. What is better ? Unique
vertex id is hard to map to a particular node in some graph in $\db$.


\begin{enumerate}
\item The database $\db$ is stored using the sparse-matrix
  representation.
\item \textbf{HOW THE EXTENSIONS $\extend(\pattern,G')$ OF A PATTERN
  $\pattern$ SHOULD BE REPRESENTED?}  I mean using ``general'' edges,
  i.e., both vertex labels + edge label or full edge information:
  vertex labels, edge labels + ids of vertices and edge. \emph{Answer:
  } the labels of edge, and both vertices is stored in the canonical
  code.
\item \textbf{how to find out that the new edge in some graph $G\in
  \db$ does not connect to other node in the pattern?} Scan the
  corresponding elements of the columns in ${\cal Q}$. Here the naive
  representation will be better !
\item \textbf{a really big problem: } the threads allocated to columns
  do not know, whether is element belongs to the embeding or not. From
  this point of view, the naive representation is better.
\end{enumerate}
\bigskip

\begin{algorithm}[!htb]
\caption{Helper function that maps thread index on two indexes: $qidx$
  representing the column index vector of embeddings $Q$ and $qidx'$
  representing the index in the $qidx$-th column}

\vbox{\textsc{ComputeIdx}(\vtop{\noindent Embeddings ${\cal Q}$, Right-most path $RMP=(r_1,\ldots,r_x)$, Integer i,
      \par\noindent (Out)Index qidx, (Out)Index qidx')}}
\begin{algorithmic}[1]
     \STATE $s\leftarrow 0$
     \STATE $qidx\leftarrow 0, qdix'\leftarrow 0$
     \FOR{$k=1,\ldots, x$}
        \IF{$s + |Q_{r_k}| > i$}
        \STATE $qidx\leftarrow r_k, qidx'\leftarrow i-s$
        \STATE \textbf{break}
        \ELSE
        \STATE  $s\leftarrow s + |Q_{r_k}|$
        \ENDIF
     \ENDFOR
\end{algorithmic}
\end{algorithm}

\begin{algorithm}[!htb]
\caption{Pseudocode of the support computation for edges on GPU}
\vbox{\textsc{GetCodeExtensions}(\vtop{\noindent Database $\db$, 
                                \par\noindent Canonical code $C$,
                                \par\noindent Embeddings ${\cal Q}$,
                                \par\noindent Right-most path ${\cal R}=(r_1,\ldots,r_x)$)}}
\begin{algorithmic}[1]
  \REQUIRE ${\cal Q} = (Q_0, Q_1, \ldots, Q_{|C|})$
%  \REQUIRE for each $G\in \db$ the set $E$ contains all embeddings, i.e., $\extend(C,G)\subseteq E$.
  \REQUIRE Right-most path of $C$ is stored in thread-local memory,
           denoted by ${\cal R}=(r_1,\ldots,r_x)$. each $r_i$ is an index into
           ${\cal Q}$.
  \STATE host computes the number of edges in the old frontier
         $f=\sum_{i=1}^x|Q_{r_i}|$ (\emph{parallel-reduction} or maybe
         sequentially on host)
  \STATE host allocates device-array $E_{\text{offsets}}$ of size $f+1$.
  \FORPARALLEL{all threads $\thread_i, 1\leq i \leq f$} 
  \STATE $qidx\leftarrow 0, qdix'\leftarrow 0$
  \STATE call \textsc{ComputeIdx}$({\cal Q}, {\cal R}, i, qidx, qidx')$ \emph{// compute the position in ${\cal Q}$ of the current thread}
  \STATE $E_{\text{offsets}}[i]\leftarrow |Q_{qidx}[qidx']|$
  \STATE all threads perform exclusive-scan operation on
         \textbf{the whole} $E_{\text{offsets}}$, filling $E_{\text{offsets}}[f+1] !$ \par(\textbf{requires
         barrier before+after scan})
  \ENDFOR
  \STATE host reads $n\leftarrow E_{\text{offsets}}[f+1]$ from device ($n$ is the size of new frontier)
  \STATE host allocates array $E_{\text{data}}$ in device-memory of size $n$
  \FORPARALLEL{all threads $\thread_i, 1\leq i \leq f$}
     \STATE $qidx\leftarrow 0, qdix'\leftarrow 0$
     \STATE call \textsc{ComputeIdx}$({\cal Q}, {\cal R}, i, qidx, qidx')$

     \STATE $k_1\leftarrow E_{\text{offsets}}[i], k_2\leftarrow E_{\text{offsets}}[i+1]$
\medskip

     \STATE store in device-global memory $E_{\text{data}}[k_1],\ldots,E_{\text{data}}[k_2-1]$
            the tuple $(l_i, l_{ij}, l_j, b, \ell, v, g)$ \par where
            $l_i,l_{ij}, l_j$ are the labels corresponding to neighborhoods
            of\par\noindent $(\ell,v,g)=Q_{qidx}[qidx']$, the $b$ is set to $1$
            if it is backward edge and to $0$ otherwise.

\medskip
     \STATE sort the array $E_{\text{data}}$, the operator $<$ is defined on the
            label and $b$ part of each array element.
\medskip

     \STATE prefix-sum segmented inclusive prefix-scan of $E_{\text{data}}$ (frontier of the
            segmented scans: if the labels+$b$ in $E_{\text{data}}[i]$ and $E_{\text{data}}[i+1]$ differs)
            1) where to store this ? 2) this should give the support of the
            extensions

  \ENDFOR
  \STATE deduplicate extensions+remove extensions(edges) that were already used by the pattern $E_{\text{data}}$. How !?
\end{algorithmic}
\end{algorithm}

