\section{Support computation}

Support computation must be split into two parts: 1) extraction of
extensions from the extension array; 2) computation of support for
each extension;

\noindent\textbf{Extract extensions:} we extract extensions
segment-by-segment. We allocate array of size $|\allvlabels|\times
|\allelabels|$ and execute $|\extensionarray_i|$ threads that store
0/1 flags in this array that represents whether a particular
combination of $(\elabel, \vlabel)\in \allvlabels\times \allelabels$
exists in the $i$-th segment.

\noindent\textbf{Support computation of extensions:} We describe
computation in one segment of $\extensionarray$. Computing supports of
extensions in all segments in parallel is similar using segmented
reduce/scan operations. From the properties of the $\extensionarray$
described in Section~\ref{sec:computing-extension} follows that in the
segments are block of vertex ids belonging to one graph and there is
one such block per segment. Therefore, by creating array of 0/1
indices by putting 1 on the end of the block. Scanning this array
results in mapping stored in array $\mathbf{M}$ of the sparse database
graph ids into a continuous range. The support computation proceeds as
follows: let have $n$ extensions and the number of graphs in the
segment is $g$. We allocate array $\mathbf{S}$ of size $n\cdot g$
allocating for $i$-th extension $i$-th block of size $g$. We then
execute $n\cdot g$ threads each assigned to an extension. Each thread
stores 1 in a block corresponding to its assigned extension at a
position looked up in $\mathbf{M}$. The final supports are then
computed by executing parallel segmented reduction in array
$\mathbf{S}$ with segments of size $g$. This can be parallelized even
more, by processing many segments at once.



\textbf{There is missing a lot of notation that make the text more
  clear and more concise. Additionally, we need to explain why we need
  to collect values $\allvlabels\times \allelabels$.}



\subsection{Growing the embeddings}

After we know the frequent extensions, we have to transform the
information from the extension array into a new column $\embedcol$ or
process some existing column. We have to consider two cases:

\textbf{(1) Forward extensions: } Forward extensions introduce new
vertex into the pattern and therefore introduce new column. The column
is extracted directly from one segment of $\extensionarray$, by
copying the needed information to a new column $\embedcol$ and
\emph{add} the new column to the embeddings.


\textbf{(2) Backward extensions: } backward extensions do not
introduce new vertex into the pattern. The backward extensions can
only filter the last column in the embeddings. There are two reason:
we consider only simple graphs and 2) the principle of
monotonicity. The last column is filtered against the \texttt{row}
field of some elements in $\extensionarray$ in the following way:
allocate a vector $\mathbf{F}$ of the same size as is the size of last
column and store there 0. Execute $|\mathbf{F}|$ threads, thread with
index $i$ stores 1 in $\mathbf{F}[i]$ if there exists a backlink from
elements of $\extensionarray$. Then execute parallel scan in
$\mathbf{F}$, which gives the new positions in the filtered
column. Finally copy the information from the last column into new
column and \emph{replace} the last column in the embeddings while
\emph{retaining} the original last column.



