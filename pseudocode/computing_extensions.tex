\section{Computing extensions}\label{sec:computing-extension}

Let have a pattern $\pattern$ that needs to be expanded. In order to
find all extension $\extend_{\db}(\pattern)$, we have to make a lookup
in the database for each embedding of the current pattern $\pattern$
and look for the extension set. Let the pattern contains $p$
vertices. We have $p$ columns $\embedcol_i$ and the right-most path
$\rmpath^{GPU}(\pattern)$. The $\rmpath^{GPU}(\pattern)$ stores vertex
ids in the pattern numbering, instead of edge ids. The embeddings are
in fact multiple linked lists with heads in the last column
$\embedcol_{p-1}$. There are $|\embedcol_{p-1}|$ embeddings that needs
to be processed. The reconstruction of the embeddings can be done only
from the last column. Each vertex on the right-most path is assigned
to one thread, having $|\rmpath^{GPU}(\pattern)| \cdot
|\embedcol_{p-1}|$. For \emph{forward extension} each thread has to
check that the new vertex is not already present in the
embedding. Vertices in the last column also has to check for
\emph{backward extensions}, i.e., it has to check that the new vertex
is present on the right-most path.  The founded extensions are stored
in the array $\extensionarray$. Each element of $\extensionarray$
contains a tuple $(\texttt{elabel}, \texttt{to},
\texttt{tolabel})$. The values of starting vertex and its label are
given by the column and the DFS code.


The threads are allocated in column-major order, i.e., threads with
numbers $1,\ldots, |\embedcol_{0}|$ are assigned to the first column,
threads $|\embedcol_{0}|+1,\ldots, |\embedcol_{1}|$ to second column,
etc.  Column-major order causes that many threads can share the same
vertex on the right-most path and therefore access the same locations
in global memory: a) the neighborhood; b) the
embeddings. \localcmnt{robert}{THIS NEEDS TO BE EXPLAINED MUCH MORE
  SOMEWHERE IN THE TEXT, maybe in the introduction to datastructures
  !!!}.  The most important feature of checking a neighborhood vertex
against the whole embedding is that this check is done
\emph{independently} by each thread, i.e., there is no
communication. Each thread allocates its right-most path vertex by
computing the assigned column and following the backlinks.


The storage of extensions is performed in four steps: 1) the maximum
degree $m$ of valid vertex extension for every vertex on the
right-most path in every embeddings is computed. Then an array
$\mathbf{V}$ of size $m\cdot |\embedcol_{p-1}| \cdot |\rmpath^{GPU}|$
for storing validity indices is allocated. The array is initially
filled with 0. 2) scan the database once and store 1 in $\mathbf{V}$,
if the extension is valid. 3) compute the prefix scan of $V$. This
step gives indices into perfectly packed extension array
$\extensionarray$. 4) store the information about extensions in
$\extensionarray$ at positions looked-up in $\mathbf{V}$.

The properties of $\extensionarray$ are the following:

\begin{enumerate}
\item Two threads with adjacent ids $i$ and $i+1$ store the embedding
  descriptions in the array in adjacent memory positions. From this
  follows an important property: the extensions are stored in segments
  that represents right-most path vertices, we denote $k$-th segment
  by $\extensionarray_k$.
\item Additionally, in each segment $\extensionarray_k$ vertices from
  one graph forms blocks inside of the segments. 
\item The last and very important property of the embedding storage is
  that its \texttt{row} pointer points to the row which the extending
  is supposed to be extend. This property is used for forming new
  embedding columns. All these properties are important for support
  computation.
\end{enumerate}

There is one important optimalization: we omit whole columns
$\embedcol_i$ from the processing, if we know beforehand that all
extensions from vertices in $\embedcol_i$ are not frequent (see the
monotonicity principle). Additionally, we have to follow the
label-based \textbf{CITE !? EXPLAIN !?} pruning of possible
extensions.





