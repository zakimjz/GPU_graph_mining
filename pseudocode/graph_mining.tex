\section{Graph mining}

In this section, we overview the GSpan algorithm~\cite{gspan}. The
input to the algorithm is a database(multiset) of graphs
$\db=\{\G_i\}, \G_i=(\V_i, \E_i)$ and a parameter
$\minsupp\in\natnum$. The GSpan algorithm outputs such graphs patterns
$\pattern=(\V_\pattern, \E_\pattern)$ that are subgraph isomorphic
with at least $\minsupp$ graphs $\G_i$ in the database. The GSpan
algorithm is the depth-first search algorithm. There are many variants
on the GSpan algorithm. We describe a variant that is suitable for our
purposes. As usually in the frequent pattern mining, the algorithm
grows the pattern by extensions that are collected from the
database. The information of pattern occurance is hold in main memory
in the form of mapping (called \emph{embedding}) from $\pattern$ into
a particular $\G_i$ in the database. All frequent pattern mining
algorithms have to find a unique string representation of the
pattern. The reason is that the frequent pattern algorithms in fact
expand this string representation by ``letters'' describing the
smallest pattern part. In our case, the letter is defined to be a
five-tuple describing one edge $(\vrtx_i, \vrtx_j, \lbl_i, \lbl_{ij},
\lbl_j)$ such that $\vrtx_i, \vrtx_j\in \V_\pattern,
(\vrtx_i,\vrtx_j)\in\E_\pattern$. A string
$S=(s_1,\ldots,s_{|\E_\pattern|}), s_\ell=(\vrtx_i, \vrtx_j, \lbl_i,
\lbl_{ij}, \lbl_j)$ that is created from $\pattern$ is called the DFS
code. For ordering relation $<$ of the DFS codes,
see~\cite{gspan}. However, for the algorithm, we need a \emph{unique
  DFS code}. Let $S',S$ are two DFS codes representing
$\pattern$. $S'$ is a \emph{unique DFS code} if and only if $S'< S$
for all $S$ such that $S\neq S'$. We denote $S'$ by
$\mindfscode(\pattern)$. The right-most path $\rmpath(\pattern)$ in
the pattern $\pattern = (\V, \E)$ are the indices of elements in
$\mindfscode(\pattern)$ representing the edges on the shortest path
from $\vrtx_0 \in \V$ to vertex $\vrtx_{|\V|} \in \V$.

The algorithm expands $\pattern$ only by edges that results in minimal
DFS code. Such approach guarantees that we do not generate one
$\pattern$ multiple times. In order to find the extensions, we have to
do a database scan. 



Our method have the same schema but it runs the two following
operations in parallel: 1) collection of the extensions; 3) support
computation; 2) expanding of the embedding. We do not execute the
minimality checking in parallel as there is not enough paralellism in
the operation.


